\documentclass{article} %article 文档
% \usepackage{ctex}  %使用宏包(为了能够显示汉字)
\title{COMP4650 Assignment 2 Answers}  %文章标题
\author{Jieli Zheng}   %作者的名称
\date{u6579712}
% 设置页面的环境,a4纸张大小,左右上下边距信息
\usepackage[a4paper,left=10mm,right=10mm,top=15mm,bottom=15mm]{geometry}
\usepackage{array}  
\usepackage{multirow}
\usepackage{booktabs}
\usepackage{graphicx}
\usepackage{float} 
\usepackage{subfigure}

\begin{document}
\maketitle

\section*{Question 1 A simple linear classifier}
Generally I make three major changes on the classic Logistic Regression model.\\
The first major change I make in preprocessor is using SnowballStemmer with 
language=english. SnowballStemmer is a nltk package built-in stemmer that can 
transform any english words back to their stems which is easier for tokenizer 
to find the right tokens from the raw text.\\
The seconde major change I make is using tokenizer to WordPunctTokenizer instead 
of WhitespaceTokenizer. The WordPunctTokenizer can split sentence into a bunch
of words and single punctunations. It is stronger than the original setting 
with WhitespaceTokenizer because WhitespaceTokenizer can't split punctunations 
from word. It definitely has different tokens like "word," "word." which should 
be the same meaning in human's perspective.\\
Last major change I make in CountVectorizer is setting the parameter 
max features to 5000. The advantage of extracting more features is that model 
can have larger observation on low frequency words with explicit tendency. 
%three changes:\\
%CountVectorizer: stop words max features\\
%Preprocessor: SnowballStemmer WordPunctTokenizer\\
%LogisticRegression: class weight\\

\section*{Question 2 Embedding based classifier}
Code in Pycharm

\section*{Question 3 Tuning a pytorch model}


\end{document}